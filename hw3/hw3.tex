\documentclass[letterpaper, 11pt]{article}
\usepackage{latexsym}
\usepackage{amssymb}
\usepackage{times}
%\usepackage[in]{fullpage}
\usepackage{amsmath,amsfonts,amsthm}
\usepackage{graphicx}

%\documentclass[11pt]{article}
%\pagestyle{myheadings}
%\usepackage[ruled,nothing]{algorithm}
%\usepackage{algorithmic}
%\usepackage[dvips]{epsfig,graphicx}
%\numberwithin{equation}{section}

\bibliographystyle{plain}

\newenvironment{newalgo}[2]{\begin{algorithm}

\caption{\textsc{#1}}\label{#2}

\begin{algorithmic}[1]}{\end{algorithmic}\end{algorithm}}



\newcommand{\gm}{\gamma}
\newcommand{\wh}{\widehat}
\newcommand{\rep}{representation}
\newcommand{\rv}{random variable}
\newcommand{\la}{\lambda}
\newcommand{\wt}{\widetilde}
\newcommand{\st}{such that}
\newcommand{\slvary}{slowly varying}
\newcommand{\ma}{moving average}
\newcommand{\regvary}{regularly varying}
\newcommand{\asy}{asymptotic}
\newcommand{\ts}{time series}
\newcommand{\id}{infinitely divisible}
\newcommand{\seq}{sequence}
\newcommand{\fidi}{finite dimensional \ds}

\newcommand{\ble}{\begin{lemma}}
\newcommand{\ele}{\end{lemma}}
\newcommand{\bfX}{{\bf X}}
\newcommand{\pro}{probabilit}
\newcommand{\BX}{{\bf X}}
\newcommand{\BY}{{\bf Y}}
\newcommand{\BZ}{{\bf Z}}
\newcommand{\BV}{{\bf V}}
\newcommand{\BW}{{\bf W}}
\newcommand{\reals}{{\mathbb R}}
\newcommand{\bbr}{\reals}

\newcommand{\balpha}{\mbox{\boldmath$\alpha$}}
\newcommand{\bbeta}{\mbox{\boldmath$\beta$}}
\newcommand{\bmu}{\mbox{\boldmath$\mu$}}
\newcommand{\tbmu}{\mbox{\boldmath${\tilde \mu}$}}
\newcommand{\bEta}{\mbox{\boldmath$\eta$}}


\def \br#1{\left \{#1 \right \}}
\def \pr#1{\left (#1 \right)}

\newcommand{\Gm}{\Gamma}
\newcommand{\ep}{\epsilon}


\newtheorem{lemma}{Lemma}[section]
\newtheorem{figur}[lemma]{Figure}
\newtheorem{theorem}[lemma]{Theorem}
\newtheorem{proposition}[lemma]{Proposition}
\newtheorem{definition}[lemma]{Definition}
\newtheorem{corollary}[lemma]{Corollary}
\newtheorem{example}[lemma]{Example}
\newtheorem{exercise}[lemma]{Exercise}
\newtheorem{remark}[lemma]{Remark}
\newtheorem{fig}[lemma]{Figure}
\newtheorem{tab}[lemma]{Table}
\newtheorem{fact}[lemma]{Fact}
\newtheorem{test}{Lemma}
\newtheorem{algorithm}[lemma]{Algorithm}

\newcommand{\play}{\displaystyle}

\newcommand{\ms}{measure}
\newcommand{\beao}{\begin{eqnarray*}}
\newcommand{\eeao}{\end{eqnarray*}\noindent}
\newcommand{\beam}{\begin{eqnarray}}
\newcommand{\eeam}{\end{eqnarray}\noindent}

\newcommand{\halmos}{\hfill\mbox{\qed}\\}
\newcommand{\fct}{function}
\newcommand{\ins}{insurance}
\newcommand{\ds}{distribution}

\newcommand{\one}{{\bf 1}}
\newcommand{\eid}{\buildrel{\rm d}\over {=}}
\newcommand {\Or}{\rm ORDER}
\newcommand {\In}{\rm INTER}

\newcommand{\bbd}{{\mathbb D}}
\newcommand{\vi}{$V_{ij}$ }
\newcommand{\rr}{R^{\prime\prime}}
%\newcommand{\R}{R^\prime}
\newcommand{\ci}{\frac{1}{c}}
\newcommand{\Vi}{V(n)}
\newcommand{\dR}{\mathcal R}
\newcommand{\md}[1]{\left(\ \rm{mod}\ \it{#1}\right)}
\newcommand{\So}{s}
%\begin{document}
%\def\DoubleSpace{\baselineskip=24pt}
%\DoubleSpace \sloppy

\begin{document}



\title{Homework \#3 \\ Introduction to Algorithms/Algorithms 1 \\ 600.363/463 }
\author{\textbf{Due on:} Friday, March 30, 11:59 a.m. (NOON)\\
\\\textbf{Where to submit:} On blackboard, under student assessment
% \\(please ask Debbie DeFord in 224 if you cannot find it)
\\ \textbf{Late submissions:} will NOT be accepted\\
\\ \textbf{Format:} Please start each problem (1 through 14) on a new page. 
\\ Please type your answers; handwritten assignments will not be accepted.
%\\ \textbf{Electronic submissions:} will NOT be accepted\\
\\ \textbf{Note that:} NUMBERED EXERCISES AND PROBLEM REFER TO THE \\ COURSE TEXTBOOK (CLRS, 3rd Edition.)\\
\\}

\maketitle

%%%%%%%%%%%%%%%%%%%%%%%%%%%%%%%%%%

\section{Adjacency Matrix}
Let G be an unweighted undirected graph with adjacency matrix $M$, and let $M_1$ be a product of the matrix by itself. i.e. $M_1 = M\times M$
\begin{itemize}
\item{What does $M_1$ represent?}
\item{What will happen if we will multiply $M$ by itself $k$ times?}
\end{itemize}
\eject

\section{Cycles} 
Design an algorithm that checks whether or not an undirected graph contains a cycle.

\section{Exercise 9.3-3} 

\section{Exercise 9.3-8}
 
\section{Problem 9-2}

\section{Exercise 22.2-4}

\section{Exercise 22.2-5}

\section{Exercise 22.2-8 (Optional)}

\section{Problem 22-1}

\section{BFS/DFS}
Explain why BFS cannot be used to perform the topological sort.
Explain why DFS cannot be used to find a shortest path.

\section{Disjoint Sets}
We know that the data structure for disjoint sets (Chapter $21$) can be implemented in such a way that $m$ operations over $n$ elements take $m \;\alpha(n)$ time 
where $\alpha(n)$ is the function from Chapter $21.4$.
Can we derive a similar bound for heaps? 
Explain your answer.

\section{MST}
Why does Kruskal's algorithm work slower then $O(E\;\alpha(E))$?
Why do we need two different algorithms for the MST?

\section{Problem 22.5-3}

\section{Problem 23.2-4}

\end{document}

%%%%%%%%%%%%%%%%%%%%%%%%%%%%%
